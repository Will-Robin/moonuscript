% Options for packages loaded elsewhere
\PassOptionsToPackage{unicode}{hyperref}
\PassOptionsToPackage{hyphens}{url}
%
\documentclass[
]{article}
\usepackage{amsmath,amssymb}
\usepackage{iftex}
\ifPDFTeX
  \usepackage[T1]{fontenc}
  \usepackage[utf8]{inputenc}
  \usepackage{textcomp} % provide euro and other symbols
\else % if luatex or xetex
  \usepackage{unicode-math} % this also loads fontspec
  \defaultfontfeatures{Scale=MatchLowercase}
  \defaultfontfeatures[\rmfamily]{Ligatures=TeX,Scale=1}
\fi
\usepackage{lmodern}
\ifPDFTeX\else
  % xetex/luatex font selection
\fi
% Use upquote if available, for straight quotes in verbatim environments
\IfFileExists{upquote.sty}{\usepackage{upquote}}{}
\IfFileExists{microtype.sty}{% use microtype if available
  \usepackage[]{microtype}
  \UseMicrotypeSet[protrusion]{basicmath} % disable protrusion for tt fonts
}{}
\makeatletter
\@ifundefined{KOMAClassName}{% if non-KOMA class
  \IfFileExists{parskip.sty}{%
    \usepackage{parskip}
  }{% else
    \setlength{\parindent}{0pt}
    \setlength{\parskip}{6pt plus 2pt minus 1pt}}
}{% if KOMA class
  \KOMAoptions{parskip=half}}
\makeatother
\usepackage{xcolor}
\usepackage{graphicx}
\makeatletter
\def\maxwidth{\ifdim\Gin@nat@width>\linewidth\linewidth\else\Gin@nat@width\fi}
\def\maxheight{\ifdim\Gin@nat@height>\textheight\textheight\else\Gin@nat@height\fi}
\makeatother
% Scale images if necessary, so that they will not overflow the page
% margins by default, and it is still possible to overwrite the defaults
% using explicit options in \includegraphics[width, height, ...]{}
\setkeys{Gin}{width=\maxwidth,height=\maxheight,keepaspectratio}
% Set default figure placement to htbp
\makeatletter
\def\fps@figure{htbp}
\makeatother
\setlength{\emergencystretch}{3em} % prevent overfull lines
\providecommand{\tightlist}{%
  \setlength{\itemsep}{0pt}\setlength{\parskip}{0pt}}
\setcounter{secnumdepth}{-\maxdimen} % remove section numbering
\ifLuaTeX
  \usepackage{selnolig}  % disable illegal ligatures
\fi
\IfFileExists{bookmark.sty}{\usepackage{bookmark}}{\usepackage{hyperref}}
\IfFileExists{xurl.sty}{\usepackage{xurl}}{} % add URL line breaks if available
\urlstyle{same}
\hypersetup{
  hidelinks,
  pdfcreator={LaTeX via pandoc}}

\author{}
\date{}

\begin{document}

\hypertarget{inserting-figures-and-captions-using-tags}{%
\section{Inserting figures and captions using
tags}\label{inserting-figures-and-captions-using-tags}}

The tag below should be replaced by \texttt{test\_img.png} from the
\texttt{Figures} directory, as defined in the \texttt{yaml} metadata
block above.

\begin{figure}
\centering
\includegraphics{Figures/test_img.png}
\caption{This is a figure caption \emph{written in markdown}. Figure
source: Hamonshū. 1, by Mori Yūzan; Yamada Geisōdō, Kyōto-shi, Meiji 36
(1903)}
\end{figure}

The tag below does not have a file associated with it, and so will not
be replaced:

\begin{figure}
\centering
\includegraphics{Figures/dead_link.png}
\caption{}
\end{figure}

The \texttt{:i} component of the tag indicates that it should be
replaced by a figure. Any other letter here will not result in a figure
insertion.

\hypertarget{examples}{%
\section{Examples}\label{examples}}

Some more examples of what happens when the tags are placed in varying
parts of the markdown are given below.

This tag is placed in an inline code field.

\texttt{\{\#f:test\_img:i\}}

This tag is placed in a code block:

\begin{verbatim}
{#f:test_img:i}
\end{verbatim}

This tag is placed in a figure field with an empty figure url.

\begin{figure}
\centering
\includegraphics{}
\caption{\{\#f:test\_img:i\}}
\end{figure}

This tag is placed in a figure field with a valid figure url.

\begin{figure}
\centering
\includegraphics{Figures/test_img.png}
\caption{\{\#f:test\_img:i\}}
\end{figure}

\begin{figure}
\centering
\includegraphics{Figures/test_img.png}
\caption{CAPTION TEST!}
\end{figure}

\end{document}
